\documentclass[12pt, a4paper]{article}
\usepackage{amsmath}
\usepackage{amssymb}
\usepackage{amsthm}

\usepackage[text={7in,10in},centering]{geometry}

\usepackage[no-math]{fontspec}
\usepackage{polyglossia}

\setdefaultlanguage{thai}
\setotherlanguage{english}

\XeTeXlinebreaklocale "th-TH"
\XeTeXlinebreakskip = 0pt plus 1pt
\setmainfont[Scale=MatchLowercase]{TH Sarabun New}
\newfontfamily{\thaifont}[Scale=MatchLowercase]{TH Sarabun New}
\renewcommand{\baselinestretch}{1.3}

\setmainfont{TH Sarabun New}
\begin{document}
%\raggedleft นางสาวชลธิชา พ่วงเฟื่อง   รหัสนักศึกษา 63090500006 \\[12pt]
\hrule\vspace{12pt}
\raggedright
\*
\begin{enumerate}
    \item การคูณด้วย 2's complement ระหว่าง (+7) $\times$ (+3)\\ $\mbox{}$ \\ 
    $A$\hspace*{2cm} $Q$ \hspace*{2cm} $Q_{-1}$ \hspace*{2cm} $M$ \hspace*{2cm}  \\  $\mbox{}$ \\ 
    0000\hspace*{2cm}0011\hspace*{2cm}0\hspace*{2cm}         0111 \hspace*{2cm}  Initial Values 
      \\  $\mbox{}$ \\ 
    1001\hspace*{2cm}0011\hspace*{2cm}0\hspace*{2cm}         0111 \hspace*{2cm}  $A$\hspace*{1cm} $A-M$
    1100\hspace*{2cm}1001\hspace*{2cm}1\hspace*{2cm}         0111 \hspace*{2cm}  Shift 
    \\  $\mbox{}$ \\ 
    1110\hspace*{2cm}0100\hspace*{2cm}1\hspace*{2cm}         0111 \hspace*{2cm}  Shift
    \\  $\mbox{}$ \\ 
    0101\hspace*{2cm}0100\hspace*{2cm}1\hspace*{2cm}         0111 \hspace*{2cm}  $A$\hspace*{1cm} $A+M$
    0010\hspace*{2cm}1010\hspace*{2cm}0\hspace*{2cm}         0111 \hspace*{2cm}  Shift
    \\  $\mbox{}$ \\ 
    0001\hspace*{2cm}0101\hspace*{2cm}0\hspace*{2cm}         0111 \hspace*{2cm}  Shift
    \\  $\mbox{}$ \\ 
    
    
    
    
    %จงหาคำตอบทั้งหมดที่เป็นจำนวนเต็มบวกของสมการ $21x+49y=903$\\ 
    % \begin{proof} สมมติให้ $A$ เป็นเซ็ต
    % \end{proof}

    % \begin{proof}
    %     \begin{enumerate}พิจารณาสมการ $21x+49y=903$ จะมีคำตอบ ก็ต่อเมื่อ $gcd(2,49)\mid 903$\\
    %         เนื่องจาก $gcd(21,49)=7$ เเละ $7\mid 903$ จะได้ว่า สมการ  $21x+49y=903$  มีคำตอบ\\
    %         โดยการเเทนค่าจะพบว่า $x_0=15$ และ $y_0=12$ เป็นคำตอบ\\
    %         ในที่นี้ $a=21$ และ $b=49$ จะได้ว่า $d=(a,b)=(21,49)=7$
    %         \\\indent\\ดังนั้น คำตอบทั้งหมดของสมการคือ\\
    %             $x=x_{0}+\frac{b}{d}t=15+\frac{49}{7}t=15+7t$\\
    %             $y=y_{0}-\frac{a}{d}t=12-\frac{21}{7}t=12-3t$    ซึ่ง $t\in \mathbb{Z}$
              
    %     \end{enumerate}
    % \end{proof}

\end{enumerate}

\end{document}